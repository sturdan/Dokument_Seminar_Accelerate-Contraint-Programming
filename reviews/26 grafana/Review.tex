\documentclass[a4paper,DIV=16]{scrartcl}

\usepackage[utf8]{inputenc}
\usepackage[T1]{fontenc}
\usepackage[german]{babel}

\usepackage[sfdefault]{FiraSans}
\usepackage{FiraMono}
\usepackage{url}
\usepackage{xcolor}
\definecolor{smdsblue}{RGB}{0, 69, 134}
\usepackage{local_style}

\pagestyle{empty}

\begin{document}
\title{Review Seminararbeit}
\date{SoSe 2024}

%% >> Bitte entsprechend Ihres EIGENEN Seminars ausfüllen! % --- 8< --- 8< ---
%8< --- \subtitle{\todo[inline]{Bitte die richtige Lehrveranstaltung
%einkommentieren}} \subtitle{Seminar Machine Learning}
\subtitle{Seminar Software Engineering für verteilte Systeme}
%\subtitle{Seminar Software Engineering in sicherheitskritischen Systemen} % ---
%>8 --- >8 --- >8 ---

% >> Bitte "\personal{...}" stehen lassen und nur den Inhalt "\todo[......" mit
% >> ihrem Namen ersetzen. Dieses Kommando wird später zur Anonymisierung
% >> genutzt. 
\author{\personal{{Daniel Sturm (1453079)}}}

\maketitle
\thispagestyle{empty}

\section*{Hinweise}
\begin{itemize}
\item Länge des Reviews: 2--3 Seiten (inklusive dieser Hinweise)
\item Jede Frage in diesem Template muss beantwortet werden! \textbf{Ersetzen}
Sie dazu die vorhandenen \texttt{\textbackslash{}todo}-Befehle im Template durch
Ihre Antworten in Fließtext oder ausführlichen Stichpunkten.
\item Die Qualität der von Ihnen verfassten Reviews geht in Ihre Gesamtnote für
das Seminar ein.
\item Entfernen Sie nach dem vollständigen Bearbeiten des Reviews das
\texttt{todonotes}-Paket im Header dieser Datei. Wenn Sie alle
\texttt{\textbackslash{}todo}-Befehle ersetzt haben, kompiliert das Dokument
weiterhin ohne Fehler.
\texttt{\textbackslash{}usepackage[color=smdsblue!25]{todonotes}}
\end{itemize}

\section*{Allgemeine Informationen}
\subsection*{Titel der zu bewertenden Arbeit}

Grafana: Sicherheitsaspekte in verteilten Systemen - Herausforderungen und
Maßnahmen

\section*{Hauptinhalt der Arbeit}
Zu Beginn der Arbeit wird ein Überblick über Grafana gegeben. Hierbei wird das
Open-Source-Tool zur Visualisierung und Überwachung von IT-Systemen vorgestellt.
Besonders hervorgehoben werden dabei die Integration von verteilten Datenquellen
über diverse Datenbanken sowie die Möglichkeit zur Erstellung von Alerts. Zum
Ende der Einleitung wird die Zielsetzung der Arbeit mit der Frage nach den
Sicherheitsaspekten von Grafana in verteilten Systemen formuliert.

Im zweiten Kapitel werden die benötigten Grundlagen über Cyber Security und
verteilte Systeme beschrieben. Hierzu wird zuerst auf das CIA-Triad sowie auf
die ISO/IEC 27001 als Basis für Informationssicherheit eingegangen. Anschließend
wird die Relevanz von Cyber Security insbesondere für Grafana, als ein
verteiltes System mit Zugriff auf viele Datenquellen, erläutert.

Im weiteren Verlauf der Arbeit werden die Sicherheitsfunktionen von Grafana
erläutert. Hierbei wird zuerst auf das Identity and Access Management (IAM)
eingegangen. Ein wichtiger Bestandteil hierbei ist der Schutz der Datenquellen
sowie der Dashboards. Die Verschlüsselung von Datenbanken sowie das Auditing
stellen weitere wichtige Sicherheitsfunktionen dar. Außerdem bietet Grafana die
Möglichkeit zum Blacklisting zur Verhinderung von DDoS-Angriffen sowie die
Möglichkeit des Abgleichs von möglicherweise geleakten Service-Tokens.
Allgemeine Sicherheitsmaßnahmen wie das regelmäßige Updaten von Grafana, Backups
und verschlüsselte Kommunikation spielen selbstverständlich auch eine wichtige
Rolle.

Zur Evaluation der Sicherheitsaspekte von Grafana wird die Open-Source-Version
mit der Cloud-/Enterprise-Version verglichen. Essenziell ist hierbei, dass
gerade bei der Nutzung durch viele Benutzer essenzielle  Sicherheitsfunktionen
wie Berechtigungen für Datenquellen, Key Management Service, Audit Logging und
automatische Updates nur in der kostenpflichtigen Version verfügbar sind.

\section*{Allgemeine Bewertung}

\subsection*{Stärken der Arbeit}

\begin{itemize}
  \item Gute Einleitung in das Thema Grafana und enges Arbeiten an der
  Grafana-Dokumentation.
  \item Gut beschrieben, warum die Sicherheit von Grafana wichtig ist (viele
  Datenquellen, DDoS, Single Point of Failure).
  \item Klar gegliederter Aufbau der Maßnahmen und Herausforderungen.
\end{itemize}


\subsection*{Schwächen der Arbeit}

\begin{itemize}
  \item Schon in der Einleitung das Ziel der Evaluation zwischen kostenloser und
  kostenpflichtiger Version beschreiben.
  \item Ist die Enterprise-Version dann sicher oder hat sie auch noch Schwächen?
  \item Penetrationstests oder Ansätze zur Umgehung von Sicherheitsmaßnahmen
  wären interessant.
  \item Genauer auf sichere Kommunikation eingehen (Protokolle wie HTTPS genauer
  erläutern, wie erfolgt der Zugriff auf die Daten, eventuell Fallbeispiel von
  verteilten Systemen einbeziehen).
\end{itemize}


\subsection*{Nutzung KI-basierter Tools}
Meine Meinung nach kam kein KI-basiertes Tools höchstens zur Rechtschreibprüfung und
Grammatikprüfung zum Einsatz, was sich aber nicht nachweisen lässt.
  

\section*{Sachliche Korrektheit}
  Die sachliche Korrektheit ist gegeben. Es wird im Wesentlichen auf Grafana
  Labs (die offzielle Dokumentation) verwiesen. Der Nutzerbericht ist in der
  Arbeit klar als solcher gekennzeichnet.

\section*{Äußere Form}
  \begin{itemize}
    \item Präzise Wortwahl: "`Auditing in Grafana bezieht sich auf den Prozess
    der Überwachung und Aufzeichnung von Aktivitäten, die in Grafana
    stattfinden."' (Seite 4)
    \item Korrekte Verwendung von Fachterminologie: "`Grafana bietet
    verschiedene Authentifizierungsmethoden, um einen sicheren Zugriff auf die
    Plattform zu gewährleisten. Dazu gehören die Authentifizierung über
    Benutzername und Passwort, SAML, OAuth und mehr."' (Seite 3)
    \item Korrekte Kommasetzung: "`Ein weiteres wesentliches Element der
    Sicherheit ist die Datenbankverschlüsselung, wobei das regelmäßige Rotieren
    der Schlüssel meist ausreichend ist."' (Seite 6)
    \item Einteilung und roter Faden sind ersichtlich und gut strukturiert:
    Vorstellung => Relevanz => Herausforderungen => Maßnahmen =>  Evaluation =>
    Schluss
    \item Angemessene Verwendung von Abbildungen: Abbildung 1 (Grafana-Beispiel)
    ist gut für die Vorstellung; Tabelle 1 ist sinnvoll für die Evaluation
    \item Abbildungen und Tabellen sind gut beschriftet
    \item Abbildungen und Tabellen werden im Fließtext referenziert
    \item Das Erscheinungsbild ist gut
  \end{itemize}
  
\end{document}
