% !TeX root = Seminararbeit.tex

% Sprachauswahl:
%  main=* setzt die Hauptsprache für das Dokument
%  - ngerman --> deutsch
%  - english --> englisch
\def\languages{main=german,english}
%\def\languages{main=english,ngerman}

% Art und Zeitpunkt des Seminars:
% - SEvS    Software Engineering für verteilte Systeme
% - ML     Machine Learning
% - SEisS    Software Engineering in sicherheitskritischen Systemen
\seminartype{SEvS}{Sommersemester 2024}

% Haupttitel der Arbeit
\title{Recent Approaches to accelerate CP Solving}
% Untertitel der Arbeit -- für Seminararbeiten nicht benötigt
% \subtitle{Concepts, Technologies, and Applications}

% Name, Matrikelnummer und E-Mail-Adresse
\author{Daniel Sturm}
\matrno{1453079}
\email{daniel.sturm@student.uni-augsburg.de}

% Transparenzangabe zur Verwendung
% künstlicher Intelligenz (KI)-basierter Tools. 
% Mögliche Optionen:

% - No          Keine KI-basierten Tools verwendet:
%               Sämtliche Inhalte sind eigenständig und ohne die
%               Unterstützung von Algorithmen oder Software, 
%               die auf KI basiert, entstanden.

% - Support     Verwendung KI-basierter Tools zur sprachlichen 
%               Verbesserung oder Korrektur des Textes:
%               Dies umfasst die Nutzung von Software zur 
%               Grammatikprüfung, Rechtschreibkorrektur 
%               und stilistischen Optimierung des Textes. 

% - Content     Verwendung KI-basierter Tools zur (teilweisen) 
%               Generierung von Inhalt:
%               Dies beinhaltet die Nutzung von KI für die Erstellung 
%               von Textabschnitten, Konzeptionierung von Ideen oder
%               Bereitstellung struktureller oder inhaltlicher Vorschläge.

\useAI{No}

% Datum der Abgabe
\date{16.06.2024}

\advisor{Tobias Foth}

%%% Local Variables:
%%% mode: latex
%%% TeX-master: "Seminararbeit"
%%% End:
