\begin{titlepage}
  \onehalfspacing
  \makeatletter
  \vspace*{1em}
  \begin{center}
    \ifdefempty{\@subject}{}{%
      {\usekomafont{subject}\@subject} \par\vspace{2em} }
      {\usekomafont{title}\@title} \ifdefempty{\@subtitle}{}{%
      \par\vspace{.5em} {\usekomafont{subtitle}\@subtitle} } \par\vspace{2em}
    \singlespacing
    {\usekomafont{author}%
      \@author\par \GetTranslation{matrno}: \@matrno\par}
    \texttt{\@email}
    \par\vspace{1.5em} {\usekomafont{publishers}\@publishers}
  \end{center}
  \makeatother

  \begin{abstract}
    \noindent%
    \paragraph*{\abstractname}
    Diese Arbeit untersucht aktuelle Ansätze zur Beschleunigung der
    Constraint-Programmierung (CP) zur Lösung von Optimierungsproblemen mit
    Nebenbedingungen. Es werden grundlegende Konzepte von Constraint
    Satisfaction Problems (CSP) und Constraint Optimization Problems (COP)
    diskutiert, gefolgt von aktuellen Beschleunigungsansätzen.

    Die untersuchten Ansätze umfassen Portfolio-Methoden, modellbasierte
    Optimierung, automatisches Parameter-Tuning und die Kombination von CP mit
    SAT-Lösungen. Leistungsbewertungen anhand von Benchmark-Problemen zeigen
    signifikante Verbesserungen, mit bis zu 70 \% weniger Auswertungen und bis zu 60%
    schnellerer Lösungsgeschwindigkeit.

    Diese Ansätze bieten wesentliche Verbesserungen in Effizienz und Wirksamkeit,
    insbesondere für große und komplexe Probleminstanzen, und unterstreichen ihre
    Bedeutung in der Constraint-Programmierung.
  \end{abstract}

  \vfill
  \centering
  \includegraphics[height=38mm]{figures/uni_siegel}
\end{titlepage}
%%% Local Variables: %% mode: latex %% TeX-master: "Seminararbeit" %% End:
