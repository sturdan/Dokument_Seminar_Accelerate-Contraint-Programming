\begin{titlepage}
  \onehalfspacing
  \makeatletter
  \vspace*{1em}
  \begin{center}
    \ifdefempty{\@subject}{}{%
      {\usekomafont{subject}\@subject} \par\vspace{2em} }
      {\usekomafont{title}\@title} \ifdefempty{\@subtitle}{}{%
      \par\vspace{.5em} {\usekomafont{subtitle}\@subtitle} } \par\vspace{2em}
    \singlespacing
    {\usekomafont{author}%
      \@author\par \GetTranslation{matrno}: \@matrno\par}
    \texttt{\@email}
    \par\vspace{1.5em} {\usekomafont{publishers}\@publishers}
  \end{center}
  \makeatother

  \begin{abstract}
    \noindent%
    \paragraph*{\abstractname}
    Constraint Programming (CP) spielt eine entscheidende Rolle bei der Lösung
    von Optimierungsproblemen mit Nebenbedingungen in verschiedenen Anwendungen
    wie Fahrzeugroutenplanung, Zeitplanung und Konfiguration. Aktuelle Ansätze
    zielen darauf ab, CP-Systeme zu beschleunigen, um schnellere Lösungen zu
    ermöglichen.

    Portfolio-Ansätze kombinieren mehrere Solver und wählen oder kombinieren sie
    dynamisch basierend auf den Problemmerkmalen aus. Die modellbasierte Optimierung
    verwendet Ersatzmodelle, während das automatische Parameter-Tuning
    Solverparameter automatisch anpasst. Die Kombination von CP mit SAT und Lazy
    Clause Generation zeigt vielversprechende Ergebnisse für die effiziente Lösung
    komplexer Probleme.

    Die Bemühungen zur Beschleunigung von CP-Systemen bieten vielfältige Strategien,
    wobei die Wirksamkeit stark von den spezifischen Problemstellungen abhängt. Die
    laufende Forschung verspricht weitere Verbesserungen, die möglicherweise von
    aufkommenden Technologien wie der Quantencomputing beeinflusst werden.
  \end{abstract}

  \vfill
  \centering
  \includegraphics[height=38mm]{figures/uni_siegel}
\end{titlepage}
%%% Local Variables: %% mode: latex %% TeX-master: "Seminararbeit" %% End:
