% !TeX root = Seminararbeit.tex

% !TeX root = Seminararbeit.tex

\documentclass[%
12pt,                % Schriftgröße
paper=a4,            % Papiergröße
captions=tableabove, % Beschriftungen für Tabellen oberhalb
]{scrartcl}

% ----------------------------------------------------
% Essential packages
% ----------------------------------------------------
\usepackage[utf8]{inputenc}
\usepackage[T1]{fontenc}

% ----------------------------------------------------
% Packages for layout adjustments
% ----------------------------------------------------

% Adjust line spacing
\usepackage{setspace}

% Publication quality tables
\usepackage{booktabs}

% ----------------------------------------------------
% Fonts
% ----------------------------------------------------
\usepackage{lmodern}
\renewcommand{\seriesdefault}{m}\selectfont

\newcommand\roboto{\fontfamily{Roboto-LF}\selectfont}
\newcommand*\robotocondensed{\roboto\fontseries{c}\selectfont}

\setkomafont{subject}{\large\robotocondensed}
\addtokomafont{title}{\LARGE}
\addtokomafont{subtitle}{\Large}
\setkomafont{author}{\normalsize\robotocondensed}
\addtokomafont{publishers}{\normalsize\robotocondensed}

% ----------------------------------------------------
% Colors
% ----------------------------------------------------
\usepackage{graphicx}
\usepackage[svgnames]{xcolor}
\definecolor{darkgreen}{rgb}{0.23,0.46,0.23}
\definecolor{smdsblue}{RGB}{0,69,134}

% ----------------------------------------------------
% Internal commands
% ----------------------------------------------------

\usepackage{etoolbox}
\makeatletter
\newcommand{\seminartype}[2]{%
  \subject{%
    Seminar\\
    \textit{\GetTranslationWarn{seminar@#1}}\\
    #2
  }
}
\newcommand{\advisor}[1]{%
  \publishers{%
    \GetTranslation{advisor}: #1\\
    \GetTranslation{institute}}
}
\newcommand{\email}[1]{\gdef\@email{#1}}
\newcommand{\matrno}[1]{\gdef\@matrno{#1}}
\newcommand{\institute}[1]{\gdef\@institute{#1}}
\newcommand{\useAI}[1]{\def\AIUse{#1}}
\makeatother

% Sprachauswahl:
%  main=* setzt die Hauptsprache für das Dokument
%  - ngerman --> deutsch
%  - english --> englisch
\def\languages{main=german,english}
%\def\languages{main=english,ngerman}

% Art und Zeitpunkt des Seminars:
% - SEvS    Software Engineering für verteilte Systeme
% - ML     Machine Learning
% - SEisS    Software Engineering in sicherheitskritischen Systemen
\seminartype{SEvS}{Sommersemester 2024}

% Haupttitel der Arbeit
\title{Bio-inspired \& Grid Computing}
% Untertitel der Arbeit -- für Seminararbeiten nicht benötigt
% \subtitle{Concepts, Technologies, and Applications}

% Name, Matrikelnummer und E-Mail-Adresse
\author{Daniel Sturm}
\matrno{1453079}
\email{daniel.sturm@student.uni-augsburg.de}

% Transparenzangabe zur Verwendung
% künstlicher Intelligenz (KI)-basierter Tools. 
% Mögliche Optionen:

% - No          Keine KI-basierten Tools verwendet:
%               Sämtliche Inhalte sind eigenständig und ohne die
%               Unterstützung von Algorithmen oder Software, 
%               die auf KI basiert, entstanden.

% - Support     Verwendung KI-basierter Tools zur sprachlichen 
%               Verbesserung oder Korrektur des Textes:
%               Dies umfasst die Nutzung von Software zur 
%               Grammatikprüfung, Rechtschreibkorrektur 
%               und stilistischen Optimierung des Textes. 

% - Content     Verwendung KI-basierter Tools zur (teilweisen) 
%               Generierung von Inhalt:
%               Dies beinhaltet die Nutzung von KI für die Erstellung 
%               von Textabschnitten, Konzeptionierung von Ideen oder
%               Bereitstellung struktureller oder inhaltlicher Vorschläge.

\useAI{No}

% Datum der Abgabe
\date{16.06.2024}

\advisor{Tobias Foth}

%%% Local Variables:
%%% mode: latex
%%% TeX-master: "Seminararbeit"
%%% End:


% ----------------------------------------------------
% Multi-lingual documents with Babel
% ----------------------------------------------------
\usepackage{csquotes}
\usepackage[\languages]{babel}

% ----------------------------------------------------
% Hyperlinks in PDF documents
% ----------------------------------------------------
\usepackage[%
bookmarks=true,         %
bookmarksopenlevel=1,   %
bookmarksopen=true,     %
bookmarksnumbered=true, %
plainpages=false,       % correct hyperlinks
pdfpagelabels=true,     % view TeX pagenumber in PDF reader
colorlinks=true,        % color highlight links
allcolors=black,        % make all links black by default
urlcolor=smdsblue,      % URL color
]{hyperref}

\makeatletter
\AtEndPreamble{
  \hypersetup{
    pdftitle=\@title,
    pdfauthor=\@author
  }
}
\makeatother

% Provides a solution to the problem with hyperref that links
% to floats actually anchor to the place below the float's caption,
% instead of anchoring to the beginning of the float
\usepackage[all]{hypcap}

% ----------------------------------------------------
% Code listings
% ----------------------------------------------------
\usepackage{listings}
\lstset{%
  frame=single,                             % Add a single line frame around listings
  frameround=ftft,                          % Rounded frame corners on top left and bottom right
  backgroundcolor=\color{gray!5},           % Slight gray shade for listings
  rulecolor=\color{black!30},               % Gray frame outline
  xleftmargin=.125\textwidth,               % Extra left margin
  xrightmargin=.125\textwidth,              % Extra right margin
  basicstyle=\small\ttfamily,               % General font style for listings
  keywordstyle=\bfseries,                   % Font style for keywords
  commentstyle=\color{gray},                % Font style for comments
  stringstyle={},                           % Font style for string literals
  numbers=left,                             % Show line numbers
  stepnumber=1,                             % Step increments for line numbers
  numberstyle={\sffamily\tiny\color{gray}}, % Font style for line numbers
  numbersep=2em,                            % Space between line numbers and code
}

% ----------------------------------------------------
% Bibliography management
% ----------------------------------------------------
\usepackage[%
backend=biber,      % Use biber to process bibliographies
natbib=true,        % Provide natbib-compatible citation commands
sorting=none,       % Sort citations by occurrence in the document
style=numeric-comp, % Use compressed numeric citations, e.g. [1-3; 5]
block=space,        % Add a little spacing inside bibliography entries
]{biblatex}
\addbibresource{literature.bib}

% Use main body font for URLs in bibliography
\urlstyle{same}

% Suppress page numbering on table of contents page(s). Works at least on one-page TOC.
\AtBeginDocument{\addtocontents{toc}{\protect\thispagestyle{empty}}} 

% Intelligent cross-referencing
% Note: Must be loaded at end of preamble (esp. after hyperref)
\usepackage{cleveref}

% ----------------------------------------------------
% Localization / translations
% ----------------------------------------------------
\usepackage{translations}

% Translations for seminar names
\NewTranslation{ngerman}{seminar@SEvS}{Software Engineering für verteilte Systeme}
\NewTranslationFallback{seminar@SEvS}{Software Engineering for Distributed Systems}
\NewTranslation{ngerman}{seminar@MS}{Machine Learning}
\NewTranslationFallback{seminar@MS}{Machine Learning}
\NewTranslation{ngerman}{seminar@SEisS}{Software Engineering in sicherheitskritischen Systemen}
\NewTranslationFallback{seminar@SEisS}{Software Engineering in Safety- and Security-Critical Systems}

% Generic translation used in template
\NewTranslation{ngerman}{advisor}{Betreuer}
\NewTranslation{ngerman}{matrno}{Matrikelnummer}
\NewTranslation{ngerman}{institute}{Softwaremethodik für verteilte Systeme (Prof. Bauer)\\Universität Augsburg}
\NewTranslation{ngerman}{regularlit}{Literatur}
\NewTranslation{ngerman}{onlinelit}{Online-Quellen}
\NewTranslation{ngerman}{honesty@title}{Eidesstattliche Erklärung}
\NewTranslation{ngerman}{honesty@body}{%
  Ich versichere, dass ich die vorliegende Arbeit ohne fremde Hilfe und ohne Benutzung anderer
  als der angegebenen Quellen angefertigt habe, und dass die Arbeit in gleicher oder ähnlicher
  Form noch keiner anderen Prüfungsbehörde vorgelegen hat.\endgraf
  Alle Ausführungen der Arbeit, die wörtlich oder sinngemäß übernommen wurden, sind als solche
  gekennzeichnet.
}
\NewTranslation{ngerman}{aiused@no}{%
  Bei der Erstellung dieses Dokuments wurde keinerlei auf künstliche Intelligenz 
  (KI)-basierte Software verwendet.
}
\NewTranslation{ngerman}{aiused@support}{%
  Bei der Erstellung dieses Dokuments wurde künstliche Intelligenz (KI)-basierte 
  Software ausschließlich zur sprachlichen Verbesserung und Korrektur verwendet. 
}
\NewTranslation{ngerman}{aiused@content}{%
  Bei der Erstellung dieses Dokuments wurde künstliche Intelligenz (KI)-basierte 
  Software zur Generierung von Inhalten verwendet. 
}


% English fallback text
\NewTranslationFallback{advisor}{Advisor}
\NewTranslationFallback{matrno}{Matriculation number}
\NewTranslationFallback{institute}{Software Methodologies for Distributed Systems (Prof. Bauer)\\University of Augsburg}
\NewTranslationFallback{regularlit}{Literature}
\NewTranslationFallback{onlinelit}{Online resources}
\NewTranslationFallback{honesty@title}{Declaration of Academic Honesty}
\NewTranslationFallback{honesty@body}{%
  Hereby, I declare that I have composed the presented paper independently on my own and without
  any other resources than the ones indicated. All thoughts taken directly or indirectly from external
  sources are properly denoted as such.\endgraf
  This paper has neither been previously submitted to another authority nor has it been published yet.
}
\NewTranslationFallback{aiused@no}{%
  No artificial intelligence (AI)-based software was used in the creation of this document.
}
\NewTranslationFallback{aiused@content}{%
  Artificial intelligence (AI)-based software was used for content generation in the 
  creation of this document.
}
\NewTranslationFallback{aiused@support}{%
Artificial intelligence (AI)-based software was used exclusively for linguistic improvement 
and correction in the creation of this document.
}


%%% Local Variables:
%%% mode: latex
%%% TeX-master: "Seminararbeit"
%%% End:


\begin{document}
\pagenumbering{roman}	
\begin{titlepage}
  \onehalfspacing
  \makeatletter
  \vspace*{1em}
  \begin{center}
    \ifdefempty{\@subject}{}{%
      {\usekomafont{subject}\@subject} \par\vspace{2em} }
      {\usekomafont{title}\@title} \ifdefempty{\@subtitle}{}{%
      \par\vspace{.5em} {\usekomafont{subtitle}\@subtitle} } \par\vspace{2em}
    \singlespacing
    {\usekomafont{author}%
      \@author\par \GetTranslation{matrno}: \@matrno\par}
    \texttt{\@email}
    \par\vspace{1.5em} {\usekomafont{publishers}\@publishers}
  \end{center}
  \makeatother

  \begin{abstract}
    \noindent%
    \paragraph*{\abstractname}
    Constraint Programming (CP) spielt eine entscheidende Rolle bei der Lösung
    von Optimierungsproblemen mit Nebenbedingungen in verschiedenen Anwendungen
    wie Fahrzeugroutenplanung, Zeitplanung und Konfiguration. Aktuelle Ansätze
    zielen darauf ab, CP-Systeme zu beschleunigen, um schnellere Lösungen zu
    ermöglichen.

    Portfolio-Ansätze kombinieren mehrere Solver und wählen oder kombinieren sie
    dynamisch basierend auf den Problemmerkmalen aus. Die modellbasierte Optimierung
    verwendet Ersatzmodelle, während das automatische Parameter-Tuning
    Solverparameter automatisch anpasst. Die Kombination von CP mit SAT und Lazy
    Clause Generation zeigt vielversprechende Ergebnisse für die effiziente Lösung
    komplexer Probleme.

    Die Bemühungen zur Beschleunigung von CP-Systemen bieten vielfältige Strategien,
    wobei die Wirksamkeit stark von den spezifischen Problemstellungen abhängt. Die
    laufende Forschung verspricht weitere Verbesserungen, die möglicherweise von
    aufkommenden Technologien wie der Quantencomputing beeinflusst werden.
  \end{abstract}

  \vfill
  \centering
  \includegraphics[height=38mm]{figures/uni_siegel}
\end{titlepage}
%%% Local Variables: %% mode: latex %% TeX-master: "Seminararbeit" %% End:


\tableofcontents

\clearpage
\pagenumbering{arabic}


\section{Einleitung: Verwendung von Constraint Programming}
\label{sec:Einleitung-Verwendung-von-Constraint-Programming}

Constraint Programming (CP) spielt in vielen modernen Anwendungen eine große
Rolle, in denen Optimierungsprobleme mit Nebenbedingungen gelöst werden müssen.
Anwendungen hierzu sind unter anderem: Vehicle Routing, Scheduling, Planung,
Konfiguration, Ressourcenallokation und Kombinatorische Optimierung. Jedes Jahr
findet die Conference on Principles and Practice of Constraint Programming
statt, in der aktuelle Forschungsergebnisse im Bereich CP diskutiert werden, was
die Wichtigkeit des Themenfeldes unterstreicht \cite{CP20we}. Neben dem
klassischen Problem der Kursplanung \cite{duboi96jo} in einer sich ständig
beschleunigenden Zeit, gibt es auch in der Industrie eine Vielzahl von
Scheduling-Problemen, wie die Zuweisung von Aufträgen zu Maschinen
\cite{gedik16jo}. Auch wurden CP-Ansätze für die Konfiguration von Netzwerken
verwendet \cite{ardisjo}. Ein weit verbreitetes kombinatorisches
Optimierungsproblem ist das 3-SAT-Problem, das beschreibt, ob eine Formel in
konjunktiver Normalform (KNF) mit Klauseln aus jeweils 3 Literalen erfüllbar ist
\cite[271]{rossi06bo}. 

Beispielsweise löst

\[ x_1=\text{false}, x_2=\text{false}, x_3=\text{false}, x_4=\text{false} \]

das folgende 3-SAT-Problem:

\[ (\lnot x_1 \lor x_2 \lor x_3) \land (x_1 \lor \lnot x_2 \lor x_4) \land (x_2
\lor x_3 \lor \lnot x_4) \]

Besonders Vehicle Routing ist in der heutigen Zeit für viele Unternehmen in der
Logistikbranche und dem flexiblen Transport von großer Bedeutung
\cite[1]{delec22jo}. Je nach Anwendungen und Problemstellung können
unterschiedliche Constraint Programming Ansätze verwendet werden. Es existieren
verschiedene Toolsets, um Constraint Programming-Probleme zu lösen. Für das
Vehicle Routing Problem, welches das Problem beschreibt, den kürzesten Gesamtweg
für \( m \) Fahrzeuge und \( n \) Orte zu finden \cite[222]{labor18jo}, kann
beispielsweise der IBM ILOG CP Optimizer verwendet werden \cite{IBMIwe}. Je nach
Problemstellung gibt es eine Vielzahl weiterer Toolsets \cite{Solviwea}.
OR-tools von Google ist eine weitere Möglichkeit zur Lösung von Constraint
Programming (CP), Linear Programming (LP), Integer Programming (IP) und Boolean
Satisfiability (SAT) Problemen \cite{ORToowe}. Geocode ist eine
Open-Source-Variante basierend auf C++ \cite{GECODwe}, und Chuffed ist eine
Variante, die \grqq lazy clause generation\grqq\ ausnutzt, um CP-Probleme
schneller zu lösen \cite{Chuff24co}. MiniZinc vereint unter anderem die zuvor
beschriebenen Softwarebibliotheken zu einer Ausdruckssprache zum Lösen einer
Vielzahl von LP, Transportproblemen (TP) und SAT-Problemen \cite{MiniZwe}. Unter
demselben Namen wird jährlich die MiniZinc Challenge ausgetragen, in der ein
Parcours von Constraint-Modellen gelöst werden muss. Am Ende werden die Solver
nach Anzahl der gelösten Modelle, der Zeit und der Qualität der Lösung bewertet
\cite{Homewe}. Gerade bei großen Problemstellungen, oder wenn, wie in der
Forschung oft üblich, mehrere Lösungen benötigt werden oder besondere Ansprüche
an die Qualität und Genauigkeit der Lösung gestellt werden, kann die Ausführung
oft lange dauern, weshalb die benötigte Zeit der Solver von großer Bedeutung
ist. Die folgende Arbeit soll aktuelle Ansätze vorstellen, mit denen CP-Probleme
schneller gelöst werden können.


\section{Grundlagen}
\label{sec:Grundlagen}
Grundsätzlich gibt es zwei Arten von Constraint-Problemen: Constraint
Satisfaction Probleme (CSP) und Constraint Optimization Probleme (COP). Ein CSP
beschreibt im Wesentlichen ein Problem, bei dem Nebenbedingungen festlegen,
welche Werte die Variablen annehmen können \cite[13]{rossi06bo}. Das Ziel
hierbei kann es entweder sein zu überprüfen, ob eine Lösung existiert, eine
mögliche Lösung zu finden oder die Menge aller möglichen Lösungen zu bestimmen.
Ein COP ist eine Erweiterung des CSP, bei dem zusätzlich eine Zielfunktion
minimiert werden muss \cite[171]{rossi06bo}. Ebenso kann es hierbei auch darum
gehen zu überprüfen, ob das Problem überhaupt lösbar ist, eine mögliche Lösung
zu finden oder die optimale Lösung zu finden.


\subsection{Constraint Satisfaction Probleme}
\label{sec:Constraint-Satisfaction-Probleme}

Ein CSP lässt sich durch ein Tripel \(P=(X,D,C)\) beschreiben, wobei \(X=\langle
x_{1},x_{2},\ldots,x_{n}\rangle\) ein n-Tupel von Variablen, \(D=\langle
D_{1},D_{2},\ldots,D_{n}\rangle\) ein n-Tupel von Domänen ist, so dass \(x_i\in
D_{i}\) erfüllt ist und \(C=\langle C_1,C_2,\ldots,C_t\rangle\) eine Menge von
Bedingungen beschreibt. Jede Bedingung \(C_j\) ist hierbei eine Untermenge des
kartesischen Produkts über \(D\). Eine Lösung für ein CSP ist ein n-Tupel
\(A=\langle a_1,a_2,\ldots,a_n\rangle \), so dass \(a_i\in D_i\) für alle \(i\)
und \(A \in C_j \quad \forall j\) \cite[16]{rossi06bo}. Ein Beispiel für ein CSP
ist das oben beschriebene 3-SAT-Problem. Das Lösen eines 3-SAT-Problems ist
NP-vollständig, was bedeutet, dass es keinen effizienten Algorithmus gibt, der
das Problem in Polynomialzeit lösen kann \cite[17]{rossi06bo}. Eine einfache
Möglichkeit, ein CSP zu lösen, ist die Verwendung von Backtracking. Hierbei wird
eine Variable nach der anderen belegt, und bei einem Widerspruch wird ein
Schritt zurückgegangen und eine andere Variable belegt \cite[85]{rossi06bo}.
Interessant ist auch, dass sich auch alle \(k>3\)-SAT-Probleme auf ein
3-SAT-Problem reduzieren lassen, wodurch auch das \(k>3\)-SAT-Problem
NP-vollständig ist \cite[206]{gritz13bo}. \mbox{K-SAT} Klauseln lassen sich nach
folgendem Schema durch Einführung einer Hilfsvariable in \mbox{K-1-SAT} Klauseln
herunterbrechen: $x_1 \vee x_2 \vee x_3 \vee x_4$ ist äquivalent zu $(x_1 \vee
x_2 \vee y) \wedge (\lnot y \vee x_3 \vee x_4)$ .Sind die Domänen der Variablen
auf boolesche Werte beschränkt, handelt es sich um ein Boolean Satisfiability
Problem (SAT).


\subsection{Constraint Optimization Probleme}
\label{sec:Constraint-Optimization-Probleme}

Viele Probleme in der realen Welt suchen nicht nur eine Lösung, sondern die
optimale Lösung. Solche Probleme lassen sich durch ein Constraint Optimization
Problem modellieren. Hierzu wird die Problemstellung um eine Zielfunktion
erweitert, die minimiert oder maximiert werden soll. Ein COP lässt sich durch
ein Tripel \(P=(X,D,C,f)\) beschreiben, wobei die ersten drei Elemente wie bei
einem CSP sind und \(f\) eine Zielfunktion ist \cite[22]{amadi15jo}. In der
Physik oder den Ingenieurwissenschaften werden COP oft auch in
Funktionsdarstellung beschrieben.

\[
\begin{aligned}
    &\text{minimiere:} \quad f(x) \\
    &\text{unter der Bedingung:} \\
    &\quad g_j(x) \leq 0 \\
    &\quad h_l(x) = 0 \\
    &\quad \underline{x_i} \leq x_i \leq \overline{x_i}
\end{aligned}
\]

Hierbei beschreiben \(g_j(x)\) und \(h_l(x)\) die Nebenbedingungen \(C\), wobei
erstere eine Ungleichungsrestriktion und letztere eine Gleichungsrestriktion
darstellt. Die letzte Gleichung beschreibt die Domäne \(D\) der Variablen
\cite[154]{marti21bo}. Je nach Art der Problemstellung unterscheiden sich die
verwendeten Algorithmen. Sind beispielsweise die Nebenbedingungen und die
Zielfunktion linear, handelt es sich um ein Linear Programming Problem (LP).
Dieses lässt sich beispielsweise mit dem Simplex-Algorithmus lösen. Das
Verfahren beruht darauf, dass die Nebenbedingungen einen n-dimensionalen
Polyeder aufspannen und die optimale Lösung auf einer Ecke des Polyeders liegt.
Der Simplex-Algorithmus sucht nun nacheinander die Ecken ab, bis die optimale
Lösung gefunden wurde. Beschränkt man sich bei den Lösungen auf ganzzahlige
Werte, spricht man von einem Integer Programming Problem (IP). Diese lassen sich
mit dem Branch-and-Bound-Algorithmus lösen \cite{dakin65jo}
\cite[99]{hofst07bo}.

\begin{figure}[h]
    \centering
    \includegraphics[width=0.5\textwidth]{figures/Simplex.PNG}
    \caption{Der Polyeder beschreibt die Nebenbedingungen, gestrichelte Linie
    die Zielfunktion, Ecken sind mögliche Lösungen \cite[100]{hofst07bo}}
    \label{fig:Simplex}
\end{figure}

Oft lassen sich Probleme jedoch nur über nichtlineare Zusammenhänge beschreiben.
Eine Möglichkeit, solche Probleme zu lösen, ist die Verwendung von
Gradientenverfahren \cite[153]{marti21bo}. Ein einfaches Gradientenverfahren zur
Minimierung von Zielfunktionen ohne Nebenbedingungen ist "gradient descent"
\cite[110]{marti21bo}. Die Idee dahinter ist es, den Gradienten, die Richtung
des steilsten Anstiegs, der Zielfunktion zu berechnen und sich je nachdem ob die
Funktion maximiert oder minimiert werden soll, in die entgegengesetzte oder
gleiche Richtung zu bewegen. Dies wird so lange wiederholt, bis ein Minimum
gefunden wurde. Eine Herausforderung hierbei ist die Wahl der Schrittweite. In
\ref*{fig:gradient} ist beispielhaft  ein Gradientenverfahren, welches in 32
Schritten terminiert, dargestellt. Ein weiteres Problem bei der Verwendung
dieser Verfahren ist auch, dass sie je nach Initialisierung oft nur lokale
Minima finden \cite[9]{boyd04bo}. Ein Ansatz, um dieses Problem zu umgehen, ist
die Verwendung von konvexen Funktionen. Dabei wird die nichtlineare Funktion
durch eine konkave Zielfunktion approximiert \cite[11]{boyd04bo}. Dies ist von
Vorteil, da konvexe Funktionen nur ein lokales Minimum haben
\cite[7]{noced06bo}. Durch die einfachere Lösbarkeit von konvexen
Optimierungsproblemen spielen sie auch in der Literatur eine wichtige Rolle
\cite[8]{boyd04bo}.

\begin{figure}[h]
    \centering
    \includegraphics[width=0.3\textwidth]{figures/[marti21bo Gradient].png}
    \caption{Beispiel eines Gradientenverfahren, welchses die Zielfunktion \\
    $f(x_1,x_2)=x_1^2+\beta x_2^2$ minimiert  \cite[112]{marti21bo}}
    \label{fig:gradient}
\end{figure}

Um auch nichtlineare Probleme mit Nebenbedingungen zu lösen, gibt es
verschiedene Ansätze welche auf den oben beschriebenen Verfahren aufbauen. Ein
Ansatz hierzu sind beispielsweise  Penalty-Methoden. Hierbei wird die
Zielfunktion um einen Strafterm $\pi(x)$ erweitert, wenn die Nebenbedingungen
verletzt werden. Optimiert wird dann die um den Strafterm erweiterte
Zielfunktion $\hat{f}(x)=f(x)+\mu\pi(x)$.
\cite[175]{marti21bo}



\section{Recent Approaches To Accelerate Constraint Programs}
\label{sec:Recent-Approaches-To-Accelerate-Constraint-Programs}


\subsection{Portfolio Ansätze}
\label{sec:Portfolio-Ansätze}

Die Idee hinter Portfolio Ansätzen ist es, mehrere Solver zu kombinieren, um so
die Performance zu steigern. 


Ein Beispiel eines Solvers, welcher in vielen SAT-Portfolio-Ansätzen verwendet
wird, ist der MiniSat Solver \cite{een04bo}. Dieser basiert im Wesentlichen auf
dem Davis-Putnam-Logemann-Loveland-Algorithmus (DPLL) \cite{davis62jo}, einer
speziellen Form des Backtracking-Algorithmus (siehe Kapitel
\ref{sec:Constraint-Satisfaction-Probleme}), der um Unit Propagation erweitert
wurde. Unit Propagation beschreibt das Belegen einer Variable, wenn sie in einer
Klausel als einzige Variable noch unbelegt ist \cite*[S.89]{rossi06bo}. Außerdem
verwendet MiniSat weitere Techniken, um die Performance zu steigern. Dazu zählen
zum Beispiel effiziente Datenstrukturen. Beispielsweise wird für jede Variable
eine Liste von Klauseln geführt, in denen sie vorkommt. Dadurch können Unit
Propagations schneller durchgeführt werden, sobald diese belegt ist, da nur über
relevante Klauseln iteriert werden muss. Um die Iteration weiter einzuschränken,
wird eine Watched-Literal-Liste geführt, in der für jede Klausel zwei
repräsentative unbelegte Literale geführt werden. Dadurch kann schneller erkannt
werden, ob die Belegung einer Variable zu einer Unit Propagation führt
\cite[505]{een04bo}. Des Weiteren wird ein Dynamic Variable Ordering verwendet,
bei dem die Literale nach ihrer Aktivität in kürzlichen Konflikten geordnet
werden. Bei der Auswahl des nächsten Literals wird dann das aktivste mögliche
Literal gewählt, um die Suche auf erfolgversprechende Pfade zu lenken. Diese Art
der Suche wird oft als Variable State Independent Decaying Sum (VSIDS) Heuristik
bezeichnet \cite[506f]{een04bo}.



Es gibt verschiedene Ansätze, wie die Solver dabei kombiniert werden können. Zum
einen besteht die Möglichkeit, die Solver statisch festzulegen, als auch
dynamisch zu wählen. Weiter besteht die Möglichkeit des Automatischen
Parameter-Tunings, bei dem die Parameter der Solver automatisch an das Problem
angepasst werden \cite[8-11]{kotth12jo}. Eine weitere Fragestellung ist die Wahl
der Solver und wann dieser eingesetzt werden soll. Der am weitesten verbreitete
Ansatz ist, einen einzelnen Solver zu Beginn der Laufzeit zu wählen und für das
ganze Problem zu verwenden. Es gibt jedoch auch Ansätze, die den Solver zur
Laufzeit wechseln oder auch mehrere Solver aus dem Portfolio gleichzeitig
verwenden \cite[11-14]{kotth12jo}. Das Hauptproblem besteht jedoch in der
richtigen Wahl des Solvers. Während in den Anfängen die Algorithmen nach
handgewählten Kriterien ausgewählt wurden, werden heute oft automatische Ansätze
verwendet. Dazu zählen Lazy Approaches, wie das Abspeichern von Fallbeispielen
oder Nearest-Neighbor Ansätze. Auch wurden schon Klassifikationen,
Entscheidungsbäume, Support Vector Machines sowie Neuronale Netze verwendet.

\begin{figure}[h]
    \centering
    \includegraphics[width=0.5\textwidth]{figures/Neuronal Nework to choose Solver [popes22jo].PNG}
    \caption{Verwendung eines Neuronalen Netzes zur Wahl eines Solvers
    \cite[105]{popes22jo}}
    \label{fig:NeuronalNework}
\end{figure}

Bei den Learning Ansätzen besteht jedoch das Problem, dass ein großer
Trainingsdatensatz benötigt wird \cite[15-16]{kotth12jo}. Eine weitere
Möglichkeit besteht darin, ein Performance-Modell der Solver zu lernen. Dadurch
kann einfach ein neuer Solver in das Portfolio aufgenommen werden, da die
Modelle für die bestehenden Solver nicht neu gelernt werden müssen
\cite[18]{kotth12jo}. Als Features für die Modelle können beispielsweise die
Anzahl der Variablen, die Anzahl der Constraints, die Domäne der Variablen oder
Verhältnisse der vorherigen Eigenschaften verwendet werden \cite[22]{kotth12jo}.
Ein Beispiel für einen Portfolio Solver ist der Sunny-Cp-Solver, welcher einen
lazy k-Nearest-Neighbor zur Auswahl eines Solver-Sets verwendet
\cite[4]{amadi15jo}.


\subsection{Model Based Optimization}
\label{sec:Model-Based-Optimization}

Sollte die objektive Funktion nicht bekannt sein und die Auswertung des
Blackbox-Modells teuer sein, kann Model Based Optimization verwendet werden, um
die Optimierung zu beschleunigen. Hierbei wird zunächst anhand einer kleinen
Anzahl von Auswertungen ein Modell der Blackbox erstellt, und dieses Modell wird
dann für die Optimierung verwendet. Auf diesem wird solange gearbeitet, bis ein
bestimmtes Budget, beispielsweise Schritte oder ein Delta-Wert, erreicht wird.
Mit diesem wird dann durch Auswertung der Blackbox das Ergebnis bestimmt,
welches entweder verwendet wird, um das Modell zu verfeinern, oder es wird als
Endresultat ausgegeben \cite[4]{bisch18pr}. 


\subsection{Automated Parameter Tuning}
\label{sec:Automated-Parameter-Tuning}

Ein weiterer Ansatz, um die Performance von Solvern zu steigern, ist das
Automated Parameter Tuning. Hierbei werden die Parameter der Solver automatisch
an das Problem angepasst:

\[
\lambda^{*} \in \arg\max p(\mathcal{A}_{\lambda}, \mathcal{D})
\]

Hierbei beschreibt \(\lambda\) die Parameter des Algorithmus, \(\mathcal{A}\)
den Algorithmus und \(\mathcal{D}\) die Domäne. \(\lambda^{*}\) beschreibt die
optimalen Parameter für das Problem. Die Funktionsweise des Parameter Tunings
ist iterativ. Grundbaustein ist ein Konfigurator, welcher die Parameter des
Algorithmus anpasst. Initial werden dem Konfigurator die Parameter und die
Domäne übergeben. Der Konfigurator probiert verschiedene Konfigurationen aus und
übergibt sie dem Zielalgorithmus. Der Algorithmus wird auf das Problem
angewendet und gibt in Form einer Kostenfunktion an, wie gut die Konfiguration
war. Der Konfigurator passt iterativ auf Basis der Kostenfunktion die Parameter
an \cite[31-38]{kotth23pr}.

\begin{figure}[h]
    \centering
    \includegraphics[width=0.5\textwidth]{figures/Automated Parameter Tuning [kotth23pr].PNG}
    \caption{Illustration des Parameter-Tuning-Prozesses \cite[34]{kotth23pr}}
    \label{fig:bild}
\end{figure}


Die Suche nach den optimalen Parametern kann entweder durch Grid Search oder
durch Random Search erfolgen. Bei Grid Search werden die Parameter systematisch
abgedeckt, während bei der Random Search die Parameter zufällig ausgewählt
werden. Grid Search ist oft ineffizient, da eine große Anzahl von unwichtigen
Parameterkombinationen getestet wird. Dafür ist es jedoch unwahrscheinlich, dass
ein relevanter Bereich ausgelassen wird. Random Search testet hingegen relevante
Parameterkombinationen zufällig, was die Effizienz deutlich erhöht. Diese
Aussage basiert auf der Annahme, dass die effektive Dimension des Problems
niedrig ist, d. h. dass nur wenige Parameter einen relevanten Einfluss auf den
Algorithmus haben. Dieser Sachverhalt ist in Abbildung
\ref{fig:GridRandomSearch} dargestellt. Bei der Grid Search werden wichtige
Parameter fixiert und unwichtige variiert, was dazu führt, dass viele unwichtige
Parameterkombinationen getestet werden. Random Search hingegen variiert
gleichzeitig alle Parameter, inklusive der wichtigen, was zu einer deutlich
effizienteren Suche unter der Annahme einer niedrigen effektiven Dimension führt
\cite[284]{bergsjo}.

\begin{figure}[h]
    \centering
    \includegraphics[width=0.5\textwidth]{figures/Grid Search vs. Random Search [kotth23pr].PNG}
    \caption{Grid Search und Random Search \cite[284]{bergsjo}}
    \label{fig:GridRandomSearch}
\end{figure}

Ein weiterer Ansatz ist Local Search, bei dem ein einzelner Parameter iterativ
angepasst wird. Hierbei wird ein einzelner Parameter ausgewählt und verändert.
Falls die Veränderung zu einer Verbesserung führt, wird der Parameter behalten,
ansonsten verworfen.

%\subsection{Qunatum Acceleratedt} \label{sec:Qunatum Accelerated}

  
\subsection{Kombination von CP und SAT}
\label{sec:Kombination-von-CP-und-SAT}

Ein weiterer Ansatz, um CP-Probleme zu beschleunigen, ist die Kombination von CP
mit SAT. Hierzu wird die sogenannte Lazy Clause Generation verwendet. Diese
Methode wandelt die Constraints eines CP-Problems in SAT-Klauseln um. Diese
Klauseln werden nicht im Voraus generiert, sondern erst, wenn sie benötigt
werden. Durch die Verwendung von SAT Solvern können die Klauseln schneller
gelöst werden. Wenn es bei der Lösung eines SAT-Problems zu einem Konflikt
kommt, wird eine Klausel dem Solver hinzugefügt. Durch diese Kombination werden
die Stärken von CP (z. B. feinkörnige Domänenreduktion) und SAT (z. B.
leistungsstarke Konfliktlösung und Backtracking) optimal genutzt, was zu einem
effizienteren und leistungsfähigeren Solver führt \cite{stuck10bo}. Ein Beispiel
für einen Solver, der diese Methode verwendet, ist der Chuffed Solver
\cite{Chuff24co}.


  
%\subsection{Decompostion methods} \label{sec:Decompostion methods}

  

\section{Schluss}
\label{sec:Schluss}
Die vorgestellten Ansätze zur Beschleunigung von Constraint Programming (CP)
bieten vielfältige Möglichkeiten, die Leistungsfähigkeit von CP-Systemen zu
verbessern und damit die Lösung komplexer Probleme effizienter zu gestalten.

Portfolio-Ansätze zeigen, dass die Kombination mehrerer Solver eine
vielversprechende Strategie ist, um die Leistung zu steigern. Die dynamische
Auswahl oder die Kombination verschiedener Solver je nach Problemstellung kann
zu verbesserten Ergebnissen führen. Model Based Optimization und Automated
Parameter Tuning bieten weitere Wege, um die Effizienz von CP zu erhöhen, indem
Modelle verwendet oder Parameter automatisch angepasst werden.

Die Kombination von CP und SAT sowie die Nutzung von Lazy Clause Generation
stellen ebenfalls vielversprechende Ansätze dar, um die Leistung von CP-Systemen
zu steigern. Durch die Integration der Stärken beider Ansätze können komplexe
Probleme effizienter gelöst werden.

Es wird jedoch deutlich, dass es keinen universellen Ansatz gibt, um CP-Probleme
zu beschleunigen, da die Effektivität der Methoden stark von der spezifischen
Problemstellung abhängt. Vielmehr ist ein ganzheitlicher Ansatz erforderlich,
der die Auswahl und Kombination verschiedener Techniken je nach Problemstellung
und Ressourcen ermöglicht.

Die fortlaufende Forschung und Entwicklung auf diesem Gebiet verspricht, die
Leistungsfähigkeit von CP-Systemen weiter zu verbessern und ihre Anwendbarkeit
auf eine Vielzahl komplexer realer Probleme zu erweitern.

Es ist anzumerken, dass die vorgestellten Ansätze nur einen Teil des breiten
Spektrums der Forschung im Bereich der Beschleunigung von Constraint Programming
darstellen. Zukünftige Innovationen könnten auch durch neue Technologien wie
Quantum Computing beeinflusst werden.

Insgesamt bleibt die Optimierung von CP-Systemen ein aktives und vielschichtiges
Forschungsfeld, das weiterhin großes Potenzial für Innovationen und Fortschritte
bietet.


% Literaturverzeichnis
\printbibliography[heading=bibintoc]

% Anhang
\include{appendix}

% Eidesstattliche Erklärung
\clearpage
\section*{\GetTranslation{honesty@title}}
\GetTranslation{honesty@body}

\vspace{2em}

\ifdefined\AIUse
    \expandafter\ifstrequal\expandafter{\AIUse}{No}{%
        \GetTranslation{aiused@no}
    }{%
        \expandafter\ifstrequal\expandafter{\AIUse}{Content}{%
            \GetTranslation{aiused@content}
        }{%
            \expandafter\ifstrequal\expandafter{\AIUse}{Support}{%
                \GetTranslation{aiused@support}
            }{% Fallback for undefined values
                \GetTranslation{aiused@no}
            }%
        }%
    }%
\else
    % Fallback, falls \useAI{} nie aufgerufen wurde
    \GetTranslation{aiused@no}
\fi

\vspace{2em}
\makeatletter
Augsburg, \@date
\par\vspace{1.5cm}
(\@author)
\makeatother

%%% Local Variables:
%%% mode: latex
%%% TeX-master: "Seminararbeit"
%%% End:


\end{document}