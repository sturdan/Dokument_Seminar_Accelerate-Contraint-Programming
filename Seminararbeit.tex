% !TeX root = Seminararbeit.tex

\documentclass[%
12pt,                % Schriftgröße
paper=a4,            % Papiergröße
captions=tableabove, % Beschriftungen für Tabellen oberhalb
]{scrartcl}

% ----------------------------------------------------
% Essential packages
% ----------------------------------------------------
\usepackage[utf8]{inputenc}
\usepackage[T1]{fontenc}

% ----------------------------------------------------
% Packages for layout adjustments
% ----------------------------------------------------

% Adjust line spacing
\usepackage{setspace}

% Publication quality tables
\usepackage{booktabs}

% ----------------------------------------------------
% Fonts
% ----------------------------------------------------
\usepackage{lmodern}
\renewcommand{\seriesdefault}{m}\selectfont

\newcommand\roboto{\fontfamily{Roboto-LF}\selectfont}
\newcommand*\robotocondensed{\roboto\fontseries{c}\selectfont}

\setkomafont{subject}{\large\robotocondensed}
\addtokomafont{title}{\LARGE}
\addtokomafont{subtitle}{\Large}
\setkomafont{author}{\normalsize\robotocondensed}
\addtokomafont{publishers}{\normalsize\robotocondensed}

% ----------------------------------------------------
% Colors
% ----------------------------------------------------
\usepackage{graphicx}
\usepackage[svgnames]{xcolor}
\definecolor{darkgreen}{rgb}{0.23,0.46,0.23}
\definecolor{smdsblue}{RGB}{0,69,134}

% ----------------------------------------------------
% Internal commands
% ----------------------------------------------------

\usepackage{etoolbox}
\makeatletter
\newcommand{\seminartype}[2]{%
  \subject{%
    Seminar\\
    \textit{\GetTranslationWarn{seminar@#1}}\\
    #2
  }
}
\newcommand{\advisor}[1]{%
  \publishers{%
    \GetTranslation{advisor}: #1\\
    \GetTranslation{institute}}
}
\newcommand{\email}[1]{\gdef\@email{#1}}
\newcommand{\matrno}[1]{\gdef\@matrno{#1}}
\newcommand{\institute}[1]{\gdef\@institute{#1}}
\newcommand{\useAI}[1]{\def\AIUse{#1}}
\makeatother

% Sprachauswahl:
%  main=* setzt die Hauptsprache für das Dokument
%  - ngerman --> deutsch
%  - english --> englisch
\def\languages{main=german,english}
%\def\languages{main=english,ngerman}

% Art und Zeitpunkt des Seminars:
% - SEvS    Software Engineering für verteilte Systeme
% - ML     Machine Learning
% - SEisS    Software Engineering in sicherheitskritischen Systemen
\seminartype{SEvS}{Sommersemester 2024}

% Haupttitel der Arbeit
\title{Bio-inspired \& Grid Computing}
% Untertitel der Arbeit -- für Seminararbeiten nicht benötigt
% \subtitle{Concepts, Technologies, and Applications}

% Name, Matrikelnummer und E-Mail-Adresse
\author{Daniel Sturm}
\matrno{1453079}
\email{daniel.sturm@student.uni-augsburg.de}

% Transparenzangabe zur Verwendung
% künstlicher Intelligenz (KI)-basierter Tools. 
% Mögliche Optionen:

% - No          Keine KI-basierten Tools verwendet:
%               Sämtliche Inhalte sind eigenständig und ohne die
%               Unterstützung von Algorithmen oder Software, 
%               die auf KI basiert, entstanden.

% - Support     Verwendung KI-basierter Tools zur sprachlichen 
%               Verbesserung oder Korrektur des Textes:
%               Dies umfasst die Nutzung von Software zur 
%               Grammatikprüfung, Rechtschreibkorrektur 
%               und stilistischen Optimierung des Textes. 

% - Content     Verwendung KI-basierter Tools zur (teilweisen) 
%               Generierung von Inhalt:
%               Dies beinhaltet die Nutzung von KI für die Erstellung 
%               von Textabschnitten, Konzeptionierung von Ideen oder
%               Bereitstellung struktureller oder inhaltlicher Vorschläge.

\useAI{No}

% Datum der Abgabe
\date{16.06.2024}

\advisor{Tobias Foth}

%%% Local Variables:
%%% mode: latex
%%% TeX-master: "Seminararbeit"
%%% End:


% ----------------------------------------------------
% Multi-lingual documents with Babel
% ----------------------------------------------------
\usepackage{csquotes}
\usepackage[\languages]{babel}

% ----------------------------------------------------
% Hyperlinks in PDF documents
% ----------------------------------------------------
\usepackage[%
bookmarks=true,         %
bookmarksopenlevel=1,   %
bookmarksopen=true,     %
bookmarksnumbered=true, %
plainpages=false,       % correct hyperlinks
pdfpagelabels=true,     % view TeX pagenumber in PDF reader
colorlinks=true,        % color highlight links
allcolors=black,        % make all links black by default
urlcolor=smdsblue,      % URL color
]{hyperref}

\makeatletter
\AtEndPreamble{
  \hypersetup{
    pdftitle=\@title,
    pdfauthor=\@author
  }
}
\makeatother

% Provides a solution to the problem with hyperref that links
% to floats actually anchor to the place below the float's caption,
% instead of anchoring to the beginning of the float
\usepackage[all]{hypcap}

% ----------------------------------------------------
% Code listings
% ----------------------------------------------------
\usepackage{listings}
\lstset{%
  frame=single,                             % Add a single line frame around listings
  frameround=ftft,                          % Rounded frame corners on top left and bottom right
  backgroundcolor=\color{gray!5},           % Slight gray shade for listings
  rulecolor=\color{black!30},               % Gray frame outline
  xleftmargin=.125\textwidth,               % Extra left margin
  xrightmargin=.125\textwidth,              % Extra right margin
  basicstyle=\small\ttfamily,               % General font style for listings
  keywordstyle=\bfseries,                   % Font style for keywords
  commentstyle=\color{gray},                % Font style for comments
  stringstyle={},                           % Font style for string literals
  numbers=left,                             % Show line numbers
  stepnumber=1,                             % Step increments for line numbers
  numberstyle={\sffamily\tiny\color{gray}}, % Font style for line numbers
  numbersep=2em,                            % Space between line numbers and code
}

% ----------------------------------------------------
% Bibliography management
% ----------------------------------------------------
\usepackage[%
backend=biber,      % Use biber to process bibliographies
natbib=true,        % Provide natbib-compatible citation commands
sorting=none,       % Sort citations by occurrence in the document
style=numeric-comp, % Use compressed numeric citations, e.g. [1-3; 5]
block=space,        % Add a little spacing inside bibliography entries
]{biblatex}
\addbibresource{literature.bib}

% Use main body font for URLs in bibliography
\urlstyle{same}

% Suppress page numbering on table of contents page(s). Works at least on one-page TOC.
\AtBeginDocument{\addtocontents{toc}{\protect\thispagestyle{empty}}} 

% Intelligent cross-referencing
% Note: Must be loaded at end of preamble (esp. after hyperref)
\usepackage{cleveref}

% ----------------------------------------------------
% Localization / translations
% ----------------------------------------------------
\usepackage{translations}

% Translations for seminar names
\NewTranslation{ngerman}{seminar@SEvS}{Software Engineering für verteilte Systeme}
\NewTranslationFallback{seminar@SEvS}{Software Engineering for Distributed Systems}
\NewTranslation{ngerman}{seminar@MS}{Machine Learning}
\NewTranslationFallback{seminar@MS}{Machine Learning}
\NewTranslation{ngerman}{seminar@SEisS}{Software Engineering in sicherheitskritischen Systemen}
\NewTranslationFallback{seminar@SEisS}{Software Engineering in Safety- and Security-Critical Systems}

% Generic translation used in template
\NewTranslation{ngerman}{advisor}{Betreuer}
\NewTranslation{ngerman}{matrno}{Matrikelnummer}
\NewTranslation{ngerman}{institute}{Softwaremethodik für verteilte Systeme (Prof. Bauer)\\Universität Augsburg}
\NewTranslation{ngerman}{regularlit}{Literatur}
\NewTranslation{ngerman}{onlinelit}{Online-Quellen}
\NewTranslation{ngerman}{honesty@title}{Eidesstattliche Erklärung}
\NewTranslation{ngerman}{honesty@body}{%
  Ich versichere, dass ich die vorliegende Arbeit ohne fremde Hilfe und ohne Benutzung anderer
  als der angegebenen Quellen angefertigt habe, und dass die Arbeit in gleicher oder ähnlicher
  Form noch keiner anderen Prüfungsbehörde vorgelegen hat.\endgraf
  Alle Ausführungen der Arbeit, die wörtlich oder sinngemäß übernommen wurden, sind als solche
  gekennzeichnet.
}
\NewTranslation{ngerman}{aiused@no}{%
  Bei der Erstellung dieses Dokuments wurde keinerlei auf künstliche Intelligenz 
  (KI)-basierte Software verwendet.
}
\NewTranslation{ngerman}{aiused@support}{%
  Bei der Erstellung dieses Dokuments wurde künstliche Intelligenz (KI)-basierte 
  Software ausschließlich zur sprachlichen Verbesserung und Korrektur verwendet. 
}
\NewTranslation{ngerman}{aiused@content}{%
  Bei der Erstellung dieses Dokuments wurde künstliche Intelligenz (KI)-basierte 
  Software zur Generierung von Inhalten verwendet. 
}


% English fallback text
\NewTranslationFallback{advisor}{Advisor}
\NewTranslationFallback{matrno}{Matriculation number}
\NewTranslationFallback{institute}{Software Methodologies for Distributed Systems (Prof. Bauer)\\University of Augsburg}
\NewTranslationFallback{regularlit}{Literature}
\NewTranslationFallback{onlinelit}{Online resources}
\NewTranslationFallback{honesty@title}{Declaration of Academic Honesty}
\NewTranslationFallback{honesty@body}{%
  Hereby, I declare that I have composed the presented paper independently on my own and without
  any other resources than the ones indicated. All thoughts taken directly or indirectly from external
  sources are properly denoted as such.\endgraf
  This paper has neither been previously submitted to another authority nor has it been published yet.
}
\NewTranslationFallback{aiused@no}{%
  No artificial intelligence (AI)-based software was used in the creation of this document.
}
\NewTranslationFallback{aiused@content}{%
  Artificial intelligence (AI)-based software was used for content generation in the 
  creation of this document.
}
\NewTranslationFallback{aiused@support}{%
Artificial intelligence (AI)-based software was used exclusively for linguistic improvement 
and correction in the creation of this document.
}


%%% Local Variables:
%%% mode: latex
%%% TeX-master: "Seminararbeit"
%%% End:


\begin{document}
\pagenumbering{roman}	
\begin{titlepage}
  \onehalfspacing
  \makeatletter
  \vspace*{1em}
  \begin{center}
    \ifdefempty{\@subject}{}{%
      {\usekomafont{subject}\@subject} \par\vspace{2em} }
      {\usekomafont{title}\@title} \ifdefempty{\@subtitle}{}{%
      \par\vspace{.5em} {\usekomafont{subtitle}\@subtitle} } \par\vspace{2em}
    \singlespacing
    {\usekomafont{author}%
      \@author\par \GetTranslation{matrno}: \@matrno\par}
    \texttt{\@email}
    \par\vspace{1.5em} {\usekomafont{publishers}\@publishers}
  \end{center}
  \makeatother

  \begin{abstract}
    \noindent%
    \paragraph*{\abstractname}
    Constraint Programming (CP) spielt eine entscheidende Rolle bei der Lösung
    von Optimierungsproblemen mit Nebenbedingungen in verschiedenen Anwendungen
    wie Fahrzeugroutenplanung, Zeitplanung und Konfiguration. Aktuelle Ansätze
    zielen darauf ab, CP-Systeme zu beschleunigen, um schnellere Lösungen zu
    ermöglichen.

    Portfolio-Ansätze kombinieren mehrere Solver und wählen oder kombinieren sie
    dynamisch basierend auf den Problemmerkmalen aus. Die modellbasierte Optimierung
    verwendet Ersatzmodelle, während das automatische Parameter-Tuning
    Solverparameter automatisch anpasst. Die Kombination von CP mit SAT und Lazy
    Clause Generation zeigt vielversprechende Ergebnisse für die effiziente Lösung
    komplexer Probleme.

    Die Bemühungen zur Beschleunigung von CP-Systemen bieten vielfältige Strategien,
    wobei die Wirksamkeit stark von den spezifischen Problemstellungen abhängt. Die
    laufende Forschung verspricht weitere Verbesserungen, die möglicherweise von
    aufkommenden Technologien wie der Quantencomputing beeinflusst werden.
  \end{abstract}

  \vfill
  \centering
  \includegraphics[height=38mm]{figures/uni_siegel}
\end{titlepage}
%%% Local Variables: %% mode: latex %% TeX-master: "Seminararbeit" %% End:


\tableofcontents

\clearpage
\pagenumbering{arabic}

\section{Einleitung}
\label{sec:Einleitung}
Lorem ipsum dolor sit amet, consectetuer adipiscing elit, sed diam nonummy nibh euismod tincidunt ut laoreet dolore magna aliquam erat volutpat. Ut wisi enim ad minim veniam, quis nostrud exerci tation ullamcorper suscipit lobortis nisl ut aliquip ex ea commodo consequat. \cite[993]{Konak2006} \cite[1--5]{Sailer2013}
% Beim letzten Zitat die Seitenangabe "1--5" nicht durch "1 \psqq" ersetzen, da hierdurch die genaue Eingrenzung des Quellennachweises verloren geht.

Lorem ipsum dolor sit amet, consectetuer adipiscing elit, sed diam nonummy nibh euismod tincidunt ut laoreet dolore magna aliquam erat volutpat. Ut wisi enim ad minim veniam, quis nostrud exerci tation ullamcorper suscipit lobortis nisl ut aliquip ex ea commodo consequat. \cite[995 \psq]{Konak2006}

\section{Hauptteil}
\label{sec:Hauptteil}

Abbildungen können z.\,B. im Unterverzeichnis \texttt{figures} abgelegt werden.
Eingebunden werden diese mit dem Befehl \texttt{\textbackslash includegraphics} innerhalb
einer \texttt{figure}-Umgebung:
\begin{figure}[htb]
  \centering
  \includegraphics[width=0.8\textwidth]{figures/Hummingbird.jpg}
  \caption{Eine Veilchenkopfelfe (auch Costakolibri genannt, vom lateinischen Calypte costae), die zur Familie der Kolibris gehört \cite{Kolibri}}
  \label{fig:kolibri}
\end{figure}

\subsection{Erste Zwischenüberschrift}
\label{sec:ErsteZwischenueberschrift}
Die Arbeit kann auch Tabellen im \texttt{table}-Environment enthalten:
\begin{table}[ht]
  \centering
  \caption{Entfernungstabelle Süddeutschland, vgl. \cite{entfernungstabelle}}
  \begin{tabular}{c r r r}
    \toprule
              & Augsburg & München & Stuttgart \\
    \midrule
    Augsburg  & -        & 61      & 149       \\
    München   & 61       & -       & 210       \\
    Stuttgart & 149      & 210     & -         \\
    \bottomrule
  \end{tabular}
  \label{tab:entfernungen}
\end{table}

\subsubsection{Erste Unterüberschrift}
\label{sec:ErsteUnterueberschrift}

Das \texttt{listings}-Paket erlaubt es, Quellcode mit Syntax-Highlighting einzubinden:

\begin{lstlisting}[language=Python,float=ht,caption={Python-Programm zur Berechnung der Fakultätsfunktion},label=lst:factorial]
def fact(n):
    """Return the n-th factorial number"""
    if n == 0:
        return 1
    else:
        return n * fact(n-1)
  
# Test output
print fact(10)
print "Done"
\end{lstlisting}

In \Cref{lst:factorial} finden Sie eine rekursive Funktion zur Fakultätsberechnung.
  
\subsection{Zweite Zwischenüberschrift}
\label{sec:ZweiteZwischenueberschrift}

TEXT

\section{Schluss}
\label{sec:Schluss}

TEXT

% Literaturverzeichnis
\printbibliography[heading=bibintoc]

% Anhang
\include{appendix}

% Eidesstattliche Erklärung
\clearpage
\section*{\GetTranslation{honesty@title}}
\GetTranslation{honesty@body}

\vspace{2em}

\ifdefined\AIUse
    \expandafter\ifstrequal\expandafter{\AIUse}{No}{%
        \GetTranslation{aiused@no}
    }{%
        \expandafter\ifstrequal\expandafter{\AIUse}{Content}{%
            \GetTranslation{aiused@content}
        }{%
            \expandafter\ifstrequal\expandafter{\AIUse}{Support}{%
                \GetTranslation{aiused@support}
            }{% Fallback for undefined values
                \GetTranslation{aiused@no}
            }%
        }%
    }%
\else
    % Fallback, falls \useAI{} nie aufgerufen wurde
    \GetTranslation{aiused@no}
\fi

\vspace{2em}
\makeatletter
Augsburg, \@date
\par\vspace{1.5cm}
(\@author)
\makeatother

%%% Local Variables:
%%% mode: latex
%%% TeX-master: "Seminararbeit"
%%% End:


\end{document}